% For help on subfiles see https://www.sharelatex.com/learn/Multi-file_LaTeX_projects
\documentclass[../main.tex]{subfile}


\begin{document}


\paragraph{} In this project we created tools for automatic static and dynamic analysis of Android application. We used androgaurd for static analysis, discussed the extracted data and how this data can be used to compare android apps. For dynamic analysis we used an extension of cuckoo called CuckooDroid. We made several bug fixes in it and made improvements by adding  an auxiliary module IntentGun and upgrading to Android 5.1. We also researched the currently known emulation detection techniques.

\paragraph{} During the course of this project we identified several areas in which future work is recommended and we will describe them here:
\paragraph{Separating API or commonly used methods from possible malicious methods} During our static analysis we extract information about every method in an APK and then calculate a signature/hash for each of these methods which can be used for finding similar methods as shown in figure \ref{fig:sonicspy_freq} and \ref{fig:sonicspy_graph}. We can see in these figures that these samples share code but its not possible to tell whether this is malicious or not without looking at it manually. We need some tools that identify the standard API methods and some other frequently used parts of code in that shared code. It will simplify our search for malicious code segments inside an app and help us calculate similarity measures based on code parts that really matters.
\paragraph{MIME type to data} As we discussed earlier that we wrote an auxiliary module "IntenGun" for cuckoodroid in order to increase code coverage. This module generates intent messages for each and every component of an Android app to activate that component. Sometimes these intent messages need to have data appended to them and currently we don't have any mechanism to add this data. This can be done checking the MIME-type in the intent-filter for that component and then appending the appropriate data according to that MIME-type.
\paragraph{Cuckoo-Droid support for higher versions of Android} Currently cuckoo-droid only supports up to Android 4.1, which is very old and needs to be upgraded. There are some challenges in achieving that and should be done in future as most of new android apps doesn't run on it.
\paragraph{Slow android emulator} We found out that running arm image of Android 5.1 or higher on Android emulator is very slow. This problem needs to be solved. One possible solution can be to use x86 image and use Houdini arm translation on top of it.
\paragraph{Anti-Emulator-detection} We concluded that there is a lot of emulator detection techniques available and it is only a matter of time that their usage picks up the pace. There are already some anti-emulator detection techniques available but this topic is not well researched and can be explored further.
\paragraph{Real life examples of Android emulator detection techniques used} We noted in the literature there is not much related to real world examples and there is a empty space between literature and whats going in the real world. It would be interesting to explore what kind of techniques real life android malware uses to avoid detection or to detect emulators. Might also be worthy to explore some non-malicious applications using these techniques to discourage reverse engineering.
\paragraph{Support of Native Code, frida} Android supports native compiled libraries but till now most of the android malware analysis tools focus on the Dex part. Developers (malicious or non malicious) had been using this as a place to hide their code from the eyes of analyzers. Therefore support for the analysis of native compiled libraries can boost up the detection capabilities of a malware analysis system and can present analyzer with more information and we recommend for future work. Frida is nice candidate for this task as it supports hooking into native functions on android.
\paragraph{Cucko-Droid on Physical devices} As we talked about emulation detection techniques, one of way of fighting it is using read devices for analysis. Real devices will also be way more faster than using an emulator and can be made to run latest android. This year (2018), there was a similar project offered in Google Summer of Code and will be most probably offered in future as well \cite{honeynet_gsoc_project}. Interested students can also apply there and do this project under the supervision of honeynet project.

\paragraph{Upgrading Cuckoo-Droid to latest cuckoo sandbox} Currently, Cuckoo-droid is on cuckoo 1.2. Recent version of cuckoo is 2.0 which way ahead of Cuckoo-Droid and is resulting in Cuckoo sandbox community distancing themselves from Cuckoo-Droid. Cuckoo-Droid should be upgraded to cuckoo 2.0 and can be a nice topic for future work.

\paragraph{Large number of methods in APKs} As compared to other executable files, android apks contains very huge number of methods and visualization or processing tools used for those other executable files format are not sufficient for APKs. Separate tools are required for APK files to visualize methods and other related meta data of an APK.

\end{document}