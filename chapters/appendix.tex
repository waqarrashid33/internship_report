% For help on subfiles see https://www.sharelatex.com/learn/Multi-file_LaTeX_projects
\documentclass[../main.tex]{subfile}

\begin{document}
	\begin{appendices}
		\chapter{Android Activity Manager tool(am)} \label{app::android_am}
		\paragraph{} Within an adb shell, you can issue commands with the activity manager (am) tool \cite{am} to perform various system actions, such as start an activity, force-stop a process, broadcast an intent, modify the device screen properties, and more. While in a shell, the syntax is:
		\begin{lstlisting}[numbers=none]
			am command
		\end{lstlisting}
		\paragraph{} You can also issue an activity manager command directly from adb without entering a remote shell. For example:
		\begin{lstlisting}[numbers=none]
			adb shell am start -a android.intent.action.VIEW
		\end{lstlisting}
		For detailed documentation on activity manager commands please visit \cite{am}.
		\paragraph{} In this project we used this tool in the IntentGun module. We pass the intent message to this tool as parameter and it starts the corresponding Activity or Service or Broadcast that intent message. Some common used commands are:

		\begin{lstlisting}[language=bash, numbers=none]
		# For Activity
		$ am start -n Package/Activity -a action -c category data
		
		# For receivers
		$ am broadcast -n Package/Receiver -a action -c category
		
		# For services
		$ am startservice -n Package/Activity -a action -c category data
		\end{lstlisting}
		
		
		\chapter{Compiling C/C++ code for android} \label{app:ndk_c++}
		Below is an example on how to compile a C program with PIE support using Android Native Development Kit (NDK).
		\paragraph{Compiling C code for Android 4.1}
		\begin{enumerate}
			\item Download and extract NDK (My NDK directory is ~/android-ndk-r16)
			\item Create a new standalone toolchain using the script ~/android-ndk-r16/build/tools/make\textunderscore standalone\textunderscore toolchain.py
			\begin{lstlisting}[language=bash, numbers=none]
				python ~/android-ndk-r16/build/tools/make_standalone_toolchain.py --arch arm --api 16 --install-dir ~/android-ndk-r16/toolchains/arm-16
			\end{lstlisting}
			\item Write your c program and compile it
			\begin{lstlisting}[language=bash]
			cd ~/android-ndk-r16/toolchains/arm-16/bin
			touch main.c
			cat > main.c
			#include<stdio.h>
			int main()
			{
				printf("Hello world!\n");
				return 0;
			}
			^C
			# Compiling
			arm-linux-androideabi-gcc main.c -o main.o
			\end{lstlisting}
			
			\item Start emulator, copy file to it, make it executable and run it
			
			\begin{lstlisting}[language=bash]
			emulator @aosx_1 -writable-system # Recently the emulator is not starting without writable, check if yours do
			adb push main.o /data/local
			adb shell chmod 06755 /data/local/main.o
			adb shell /data/local/main.o
			
			Hello world!
			\end{lstlisting}
		\end{enumerate}		


		\paragraph{Compiling C code for Android 5.1+}
		\begin{enumerate}
			\item Download and extract NDK (My NDK directory is ~/android-ndk-r16)
			\item Create a new standalone toolchain using the script ~/android-ndk-r16/build/tools/make\textunderscore standalone\textunderscore toolchain.py
			\begin{lstlisting}[language=bash, numbers=none]
			python ~/android-ndk-r16/build/tools/make_standalone_toolchain.py --arch x86 --api 22 --install-dir ~/android-ndk-r16/toolchains/x86-22
			\end{lstlisting}
			\item Write your c program, compile it and link it separately with required flags for PIE.
			\begin{lstlisting}[language=bash]
			cd ~/android-ndk-r16/toolchains/x86-22/bin
			touch main.c
			cat > main.c
			
			#include<stdio.h>
			int main()
			{
			printf("Hello world!\n");
			return 0;
			}
			^C
			# compiling and linking
			./i686-linux-android-gcc -c main.c  -fPIE
			./i686-linux-android-gcc -lm main.o -pie
			\end{lstlisting}
			\item Start emulator, copy file to it, make it executable and run it
			\begin{lstlisting}[language=bash, firstnumber=15]
			emulator @Nexus_One_API_22_x86 -writable-system
			adb push a.out /data/local
			adb shell chmod 06755 /data/local/a.o
			adb shell /data/local/a.o
			WARNING: linker: /data/local/a.out: unused DT entry: type 0x6ffffffe arg 0x32c
			WARNING: linker: /data/local/a.out: unused DT entry: type 0x6fffffff arg 0x1
			Hello world!
			\end{lstlisting}
		\end{enumerate}
	
	\chapter{Malwares that uses Anti-Emulator detection}
	The purpose of this list is to help other researchers who are looking for samples that uses anit-emulation detection.
	\begin{itemize}
		\item 0ef01f50b829355b73418adcb30b20a3a1f6cb95123afce2c5469cdb25c7f7d0 Skinner
		\item 9AB5A05BC3C8F1931A3A49278E18D2116F529704 (Android/TrojanDropper.Agent.BKY) Uses delayed execution \cite{eset_multi_stage_malware}
		\item 2E47C816A517548A0FBF809324D63868708D00D0 (Android/TrojanDropper.Agent.BKY) Uses delayed execution \cite{eset_multi_stage_malware}
		\item DE64139E6E91AC0DDE755D2EF49D60251984652F (Android/TrojanDropper.Agent.BKY) Uses delayed execution \cite{eset_multi_stage_malware}
		\item 6AB844C8FD654AAEC29DAC095214F4430012EE0E (Android/TrojanDropper.Agent.BKY) Uses delayed execution \cite{eset_multi_stage_malware}
		\item C8DD6815F30367695938A7613C11E029055279A2 (Android/TrojanDropper.Agent.BKY) Uses delayed execution \cite{eset_multi_stage_malware}
		\item 47442BFDFBC0FB350B8B30271C310FE44FFB119A (Android/TrojanDropper.Agent.BKY) Uses delayed execution \cite{eset_multi_stage_malware}
		\item 604E6DCDF1FA1F7B5A85892AC3761BED81405BF6 (Android/TrojanDropper.Agent.BKY) Uses delayed execution \cite{eset_multi_stage_malware}
		\item 532079B31E3ACEF2D71C75B31D77480304B2F7B9 (Android/TrojanDropper.Agent.BKY) Uses delayed execution \cite{eset_multi_stage_malware}
	\end{itemize}
\end{appendices}

\end{document}
