% For help on subfiles see https://www.sharelatex.com/learn/Multi-file_LaTeX_projects
\documentclass[../main.tex]{subfile}


\begin{document}


		There are several static analysis tools available for APKs, each one having its own strengths and weaknesses.
		\todo[inline]{Add some info about common tools}
		
		
		\lipsum[1]
		\subsection{Apktool}\label{sec:apktool}
		APKTool is one of the major reverse engineering tool for android applications.  \todo[inline]{Add more info}
		\lipsum[2]
		\subsection{Androguard}\label{sec:androguard}
		\todo[inline]{Introduce androguard}
		Androguard is an open source tool written in python for analyzing android applications. Its been in a several of tools including Virustotal and Cuckoodroid among others. It can process APK files, dex files or odex files. It can disassemble Dex/Odex files to smali code and can decompile Dex/Odex to Java code. The classes in androguard can be generally divided into two categories: Classes used for parsing and the analysis classes. We will go into more details about these classes but first we will show some basic usage of androguard.
		
		
		
		\todo[inline]{TODO: Do androguard basic usage examples}
		\lipsum[1]
		\todo[inline]{Discuss the changes we made including normalization, canonical hasing for similarity search}
		\lipsum[1]
		\todo[inline]{Discuss the info we are extracting from apks for platform}
		\lipsum[1]
		\todo[inline]{TODO: Do androguard comparison apks to see how many functions has added and how many removed, make a
			table out of it}
		\lipsum[1]
		\todo[inline]{TODO: Find reused code section in sonicspy or bankbots or lokibot}
		\lipsum[1]
		\todo[inline]{Usage of androguard for extracting features for AI/ML, prepare for talk in AIOLI-FFM group}
		\lipsum[1]
		\todo[inline]{Ask lukas for some results from platform}
		\lipsum[1]
		\todo[inline]{Improvements in androguard}
		\lipsum[1]
\end{document}